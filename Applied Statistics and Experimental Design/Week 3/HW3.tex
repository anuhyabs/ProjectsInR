% Options for packages loaded elsewhere
\PassOptionsToPackage{unicode}{hyperref}
\PassOptionsToPackage{hyphens}{url}
%
\documentclass[
]{article}
\usepackage{lmodern}
\usepackage{amssymb,amsmath}
\usepackage{ifxetex,ifluatex}
\ifnum 0\ifxetex 1\fi\ifluatex 1\fi=0 % if pdftex
  \usepackage[T1]{fontenc}
  \usepackage[utf8]{inputenc}
  \usepackage{textcomp} % provide euro and other symbols
\else % if luatex or xetex
  \usepackage{unicode-math}
  \defaultfontfeatures{Scale=MatchLowercase}
  \defaultfontfeatures[\rmfamily]{Ligatures=TeX,Scale=1}
\fi
% Use upquote if available, for straight quotes in verbatim environments
\IfFileExists{upquote.sty}{\usepackage{upquote}}{}
\IfFileExists{microtype.sty}{% use microtype if available
  \usepackage[]{microtype}
  \UseMicrotypeSet[protrusion]{basicmath} % disable protrusion for tt fonts
}{}
\makeatletter
\@ifundefined{KOMAClassName}{% if non-KOMA class
  \IfFileExists{parskip.sty}{%
    \usepackage{parskip}
  }{% else
    \setlength{\parindent}{0pt}
    \setlength{\parskip}{6pt plus 2pt minus 1pt}}
}{% if KOMA class
  \KOMAoptions{parskip=half}}
\makeatother
\usepackage{xcolor}
\IfFileExists{xurl.sty}{\usepackage{xurl}}{} % add URL line breaks if available
\IfFileExists{bookmark.sty}{\usepackage{bookmark}}{\usepackage{hyperref}}
\hypersetup{
  pdftitle={DATA 557 Homework Assignment 3},
  pdfauthor={Anuhya B S},
  hidelinks,
  pdfcreator={LaTeX via pandoc}}
\urlstyle{same} % disable monospaced font for URLs
\usepackage[margin=1in]{geometry}
\usepackage{color}
\usepackage{fancyvrb}
\newcommand{\VerbBar}{|}
\newcommand{\VERB}{\Verb[commandchars=\\\{\}]}
\DefineVerbatimEnvironment{Highlighting}{Verbatim}{commandchars=\\\{\}}
% Add ',fontsize=\small' for more characters per line
\usepackage{framed}
\definecolor{shadecolor}{RGB}{248,248,248}
\newenvironment{Shaded}{\begin{snugshade}}{\end{snugshade}}
\newcommand{\AlertTok}[1]{\textcolor[rgb]{0.94,0.16,0.16}{#1}}
\newcommand{\AnnotationTok}[1]{\textcolor[rgb]{0.56,0.35,0.01}{\textbf{\textit{#1}}}}
\newcommand{\AttributeTok}[1]{\textcolor[rgb]{0.77,0.63,0.00}{#1}}
\newcommand{\BaseNTok}[1]{\textcolor[rgb]{0.00,0.00,0.81}{#1}}
\newcommand{\BuiltInTok}[1]{#1}
\newcommand{\CharTok}[1]{\textcolor[rgb]{0.31,0.60,0.02}{#1}}
\newcommand{\CommentTok}[1]{\textcolor[rgb]{0.56,0.35,0.01}{\textit{#1}}}
\newcommand{\CommentVarTok}[1]{\textcolor[rgb]{0.56,0.35,0.01}{\textbf{\textit{#1}}}}
\newcommand{\ConstantTok}[1]{\textcolor[rgb]{0.00,0.00,0.00}{#1}}
\newcommand{\ControlFlowTok}[1]{\textcolor[rgb]{0.13,0.29,0.53}{\textbf{#1}}}
\newcommand{\DataTypeTok}[1]{\textcolor[rgb]{0.13,0.29,0.53}{#1}}
\newcommand{\DecValTok}[1]{\textcolor[rgb]{0.00,0.00,0.81}{#1}}
\newcommand{\DocumentationTok}[1]{\textcolor[rgb]{0.56,0.35,0.01}{\textbf{\textit{#1}}}}
\newcommand{\ErrorTok}[1]{\textcolor[rgb]{0.64,0.00,0.00}{\textbf{#1}}}
\newcommand{\ExtensionTok}[1]{#1}
\newcommand{\FloatTok}[1]{\textcolor[rgb]{0.00,0.00,0.81}{#1}}
\newcommand{\FunctionTok}[1]{\textcolor[rgb]{0.00,0.00,0.00}{#1}}
\newcommand{\ImportTok}[1]{#1}
\newcommand{\InformationTok}[1]{\textcolor[rgb]{0.56,0.35,0.01}{\textbf{\textit{#1}}}}
\newcommand{\KeywordTok}[1]{\textcolor[rgb]{0.13,0.29,0.53}{\textbf{#1}}}
\newcommand{\NormalTok}[1]{#1}
\newcommand{\OperatorTok}[1]{\textcolor[rgb]{0.81,0.36,0.00}{\textbf{#1}}}
\newcommand{\OtherTok}[1]{\textcolor[rgb]{0.56,0.35,0.01}{#1}}
\newcommand{\PreprocessorTok}[1]{\textcolor[rgb]{0.56,0.35,0.01}{\textit{#1}}}
\newcommand{\RegionMarkerTok}[1]{#1}
\newcommand{\SpecialCharTok}[1]{\textcolor[rgb]{0.00,0.00,0.00}{#1}}
\newcommand{\SpecialStringTok}[1]{\textcolor[rgb]{0.31,0.60,0.02}{#1}}
\newcommand{\StringTok}[1]{\textcolor[rgb]{0.31,0.60,0.02}{#1}}
\newcommand{\VariableTok}[1]{\textcolor[rgb]{0.00,0.00,0.00}{#1}}
\newcommand{\VerbatimStringTok}[1]{\textcolor[rgb]{0.31,0.60,0.02}{#1}}
\newcommand{\WarningTok}[1]{\textcolor[rgb]{0.56,0.35,0.01}{\textbf{\textit{#1}}}}
\usepackage{graphicx,grffile}
\makeatletter
\def\maxwidth{\ifdim\Gin@nat@width>\linewidth\linewidth\else\Gin@nat@width\fi}
\def\maxheight{\ifdim\Gin@nat@height>\textheight\textheight\else\Gin@nat@height\fi}
\makeatother
% Scale images if necessary, so that they will not overflow the page
% margins by default, and it is still possible to overwrite the defaults
% using explicit options in \includegraphics[width, height, ...]{}
\setkeys{Gin}{width=\maxwidth,height=\maxheight,keepaspectratio}
% Set default figure placement to htbp
\makeatletter
\def\fps@figure{htbp}
\makeatother
\setlength{\emergencystretch}{3em} % prevent overfull lines
\providecommand{\tightlist}{%
  \setlength{\itemsep}{0pt}\setlength{\parskip}{0pt}}
\setcounter{secnumdepth}{-\maxdimen} % remove section numbering

\title{DATA 557 Homework Assignment 3}
\author{Anuhya B S}
\date{February 5, 2022}

\begin{document}
\maketitle

\hypertarget{data-lead.csv}{%
\section{Data: `lead.csv'}\label{data-lead.csv}}

The data are from a study of the association between exposure to lead
and IQ. The study was conducted in an urban area around a lead smelter.
A random sample of 124 children who lived in the area was selected. Each
study participant had a blood sample drawn in both 1972 and 1973 to
assess blood concentrations of lead. The children were grouped based on
their blood concentrations as follows:

Group 1: concentration \textless{} 40 mg/L in both 1972 and 1973 Group
2: concentration \textgreater{} 40 mg/L in both 1972 and 1973 or
\textgreater{} 40 mg/L in 1973 alone (3 participants) Group 3:
concentration \textgreater{} 40 mg/L in 1972 but \textless{} 40 mg/L in
1973

Each participant completed an IQ test in 1973. (A subset of the IQ
scores from this study were used in HW 1, Question 3.) The variables in
the data set are listed below.

ID: Participant identification number SEX: Participant sex (1=M or 2=F)
GROUP: As described above (1, 2, or 3) IQ: IQ score

\begin{Shaded}
\begin{Highlighting}[]
\NormalTok{leadData <-}\StringTok{ }\KeywordTok{read.csv}\NormalTok{(}\StringTok{'lead_study.csv'}\NormalTok{)}
\NormalTok{leadData}
\end{Highlighting}
\end{Shaded}

\begin{verbatim}
##     SEX GROUP  IQ
## 1     1     1  70
## 2     1     1  85
## 3     1     1  86
## 4     1     1  76
## 5     1     1  84
## 6     1     1  96
## 7     1     1  94
## 8     1     1  97
## 9     1     1  99
## 10    1     1  80
## 11    1     1 118
## 12    1     1  86
## 13    1     1 141
## 14    1     1  96
## 15    1     1  96
## 16    1     1 107
## 17    1     1  80
## 18    1     1 107
## 19    1     1 101
## 20    1     1  96
## 21    1     1  99
## 22    1     1  99
## 23    1     1 105
## 24    1     1  50
## 25    1     1 120
## 26    1     1  93
## 27    1     1 100
## 28    1     1 105
## 29    1     1  80
## 30    1     1 111
## 31    1     1 104
## 32    1     1  85
## 33    1     1  94
## 34    1     1  75
## 35    1     1  76
## 36    1     1 107
## 37    1     1  88
## 38    1     1 107
## 39    1     1  85
## 40    1     1  76
## 41    1     1  95
## 42    1     1  86
## 43    1     1  89
## 44    1     1  76
## 45    1     1 101
## 46    1     1  74
## 47    1     2  82
## 48    1     2  85
## 49    1     2  75
## 50    1     2  85
## 51    1     2 101
## 52    1     2  94
## 53    1     2  88
## 54    1     2 104
## 55    1     2  88
## 56    1     2 112
## 57    1     2  83
## 58    1     2 104
## 59    1     2 101
## 60    1     2  96
## 61    1     2  76
## 62    1     2  80
## 63    1     2  79
## 64    1     3  72
## 65    1     3  92
## 66    1     3  86
## 67    1     3  79
## 68    1     3  83
## 69    1     3 114
## 70    1     3  93
## 71    1     3  98
## 72    1     3  46
## 73    1     3  82
## 74    1     3  92
## 75    1     3 111
## 76    1     3  78
## 77    2     1  56
## 78    2     1 115
## 79    2     1  77
## 80    2     1 128
## 81    2     1  88
## 82    2     1  86
## 83    2     1  91
## 84    2     1 125
## 85    2     1 115
## 86    2     1 106
## 87    2     1  96
## 88    2     1  99
## 89    2     1  85
## 90    2     1  88
## 91    2     1  87
## 92    2     1  98
## 93    2     1  78
## 94    2     1  87
## 95    2     1  94
## 96    2     1  89
## 97    2     1  73
## 98    2     1  89
## 99    2     1  96
## 100   2     1  72
## 101   2     1  97
## 102   2     1  76
## 103   2     1 104
## 104   2     1  96
## 105   2     1 108
## 106   2     1 102
## 107   2     1  77
## 108   2     1  92
## 109   2     2  93
## 110   2     2  80
## 111   2     2  89
## 112   2     2  80
## 113   2     2  88
## 114   2     2  92
## 115   2     2  75
## 116   2     3  90
## 117   2     3  71
## 118   2     3 100
## 119   2     3  91
## 120   2     3  91
## 121   2     3  85
## 122   2     3  97
## 123   2     3  91
## 124   2     3  77
## 125   2     1 111
## 126   2     1  79
## 127   2     1 124
## 128   2     1  89
## 129   2     1  85
## 130   2     1  88
## 131   2     1 120
## 132   2     1 113
## 133   2     1 100
## 134   2     2  92
## 135   2     2  79
## 136   2     2  87
## 137   2     2  80
## 138   2     2  86
## 139   2     2  91
## 140   2     2  73
## 141   2     2  87
## 142   2     3  70
## 143   2     3  94
## 144   2     3  92
## 145   2     3  90
## 146   2     3  86
## 147   2     3  95
## 148   2     3  88
## 149   2     3  75
\end{verbatim}

\textbf{1. The first goal is to compare the mean IQ scores for males and
females. Use a 2-sample t-test for this comparison. What is the
p-value?}

For the goal to compare the mean IQ scores, the null hypothesis is:
\[H_0: \mu_M = \mu_F\] where \(\mu_M\) is the mean IQ score for male and
\(\mu_F\) is the mean IQ score for female.

\begin{Shaded}
\begin{Highlighting}[]
\NormalTok{m =}\StringTok{ }\KeywordTok{with}\NormalTok{(leadData,}\KeywordTok{tapply}\NormalTok{(IQ, SEX, mean))}
\NormalTok{s =}\StringTok{ }\KeywordTok{with}\NormalTok{(leadData,}\KeywordTok{tapply}\NormalTok{(IQ, SEX, sd))}
\NormalTok{n =}\StringTok{ }\KeywordTok{with}\NormalTok{(leadData,}\KeywordTok{tapply}\NormalTok{(IQ, SEX, length))}
\KeywordTok{data.frame}\NormalTok{(m,s,n,(s}\OperatorTok{^}\DecValTok{2}\NormalTok{))}
\end{Highlighting}
\end{Shaded}

\begin{verbatim}
##          m        s  n   X.s.2.
## 1 91.23684 14.93083 76 222.9298
## 2 90.87671 13.58507 73 184.5540
\end{verbatim}

Since the variances are nearly equal, we consider the 2 sample equal
variance T-test.

\begin{Shaded}
\begin{Highlighting}[]
\KeywordTok{t.test}\NormalTok{(leadData}\OperatorTok{$}\NormalTok{IQ[leadData}\OperatorTok{$}\NormalTok{SEX}\OperatorTok{==}\DecValTok{1}\NormalTok{],leadData}\OperatorTok{$}\NormalTok{IQ[leadData}\OperatorTok{$}\NormalTok{SEX}\OperatorTok{==}\DecValTok{2}\NormalTok{],}\DataTypeTok{var.equal =}\NormalTok{ T)}
\end{Highlighting}
\end{Shaded}

\begin{verbatim}
## 
##  Two Sample t-test
## 
## data:  leadData$IQ[leadData$SEX == 1] and leadData$IQ[leadData$SEX == 2]
## t = 0.15381, df = 147, p-value = 0.878
## alternative hypothesis: true difference in means is not equal to 0
## 95 percent confidence interval:
##  -4.267092  4.987351
## sample estimates:
## mean of x mean of y 
##  91.23684  90.87671
\end{verbatim}

The p-value is \textbf{0.878}.

\textbf{2. State the conclusion from your test.} Since the p-value is
greater than the level of significance 0.05, we \textbf{do not have
enough evidence to reject the hull hypothesis} of equal mean IQ scores
for males and females.

\textbf{3. Are the independence assumptions valid for the t-test in this
situation? Give a brief explanation.}

The independence assumptions are valid for the t-test in this case as
the group of male and female are independent of each other. The data
collected is not paired so the data collected from one group would not
have an effect on the other group.

\textbf{4. The second goal is to compare the mean IQ scores in the 3
groups. State in words the null hypothesis for this test.}

The null hypothesis is: \[H_0: \mu_1 = \mu_2 = \mu_3\] where \(\mu_1\),
\(\mu_2\), \(\mu_3\) are the mean IQ scores of group 1, 2 and 3
respectively. The null hypothesis is that the mean IQ scores are all
equal for groups 1, 2 and 3.

\textbf{5. State in words the alternative hypothesis for this test.}

The alternative hypothesis is that the mean IQ scores are not all equal
for groups 1, 2 and 3.

\textbf{6. What method should be used to perform the test?}

\begin{Shaded}
\begin{Highlighting}[]
\NormalTok{m =}\StringTok{ }\KeywordTok{with}\NormalTok{(leadData,}\KeywordTok{tapply}\NormalTok{(IQ, GROUP, mean))}
\NormalTok{s =}\StringTok{ }\KeywordTok{with}\NormalTok{(leadData,}\KeywordTok{tapply}\NormalTok{(IQ, GROUP, sd))}
\NormalTok{n =}\StringTok{ }\KeywordTok{with}\NormalTok{(leadData,}\KeywordTok{tapply}\NormalTok{(IQ, GROUP, length))}
\KeywordTok{data.frame}\NormalTok{(m,s,n,(s}\OperatorTok{^}\DecValTok{2}\NormalTok{))}
\end{Highlighting}
\end{Shaded}

\begin{verbatim}
##          m         s  n    X.s.2.
## 1 93.72414 15.570313 87 242.43464
## 2 87.65625  9.502493 32  90.29738
## 3 86.96667 12.962767 30 168.03333
\end{verbatim}

The most appropriate test to perform the comparison of two or more
groups is the \textbf{ANOVA test}. In this case, we use the ANOVA test
to test equal mean IQ scores in all 3 groups.

\textbf{7. Perform the test. Report the p-value.}

\begin{Shaded}
\begin{Highlighting}[]
\KeywordTok{summary}\NormalTok{(}\KeywordTok{aov}\NormalTok{(leadData}\OperatorTok{$}\NormalTok{IQ}\OperatorTok{~}\NormalTok{leadData}\OperatorTok{$}\NormalTok{GROUP))}
\end{Highlighting}
\end{Shaded}

\begin{verbatim}
##                 Df Sum Sq Mean Sq F value Pr(>F)  
## leadData$GROUP   1   1321  1320.6   6.766 0.0102 *
## Residuals      147  28692   195.2                 
## ---
## Signif. codes:  0 '***' 0.001 '**' 0.01 '*' 0.05 '.' 0.1 ' ' 1
\end{verbatim}

The p-value is \textbf{0.0102}.

\textbf{8. State your conclusion about the evidence for an association
between lead exposure and IQ.}

We \textbf{reject the null hypothesis} of an association between lead
exposure and IQ as the p-value is much smaller than the value of level
of significance (0.05).

\textbf{9. Are there strong reasons to believe that the assumptions of
this test are not met? Briefly justify your answer.}

ALl the following assumptions must be met for ANOVA test: 1.
Independence (of samples and of observations within each sample) 2.
Equal variances 3. Large sample sizes or normal distributions

All the three groups do not have equal variances (242,90,168 - seen as a
part of Q7). Thus there is not strong reason to believe the assumption
of the test is not met. We can assume that the data for Group 1, Group 2
and Group 3 are independent of each other, however since not all the
assumptions of ANOVA test are met, the F-test may not have the right
type I error probability.

\textbf{10. Conduct all pairwise comparison of group means. Report the
p-values.}

Since the variances are not relatively equal for all three groups, we
would perform the Welch t-test for the pairwise comparison.

\begin{Shaded}
\begin{Highlighting}[]
\NormalTok{p12 =}\StringTok{ }\KeywordTok{t.test}\NormalTok{(leadData}\OperatorTok{$}\NormalTok{IQ[leadData}\OperatorTok{$}\NormalTok{GROUP}\OperatorTok{==}\StringTok{'1'}\NormalTok{],leadData}\OperatorTok{$}\NormalTok{IQ[leadData}\OperatorTok{$}\NormalTok{GROUP}\OperatorTok{==}\StringTok{'2'}\NormalTok{],}\DataTypeTok{var.equal =}\NormalTok{ F)}
\NormalTok{p12}
\end{Highlighting}
\end{Shaded}

\begin{verbatim}
## 
##  Welch Two Sample t-test
## 
## data:  leadData$IQ[leadData$GROUP == "1"] and leadData$IQ[leadData$GROUP == "2"]
## t = 2.5622, df = 90.607, p-value = 0.01205
## alternative hypothesis: true difference in means is not equal to 0
## 95 percent confidence interval:
##   1.363464 10.772312
## sample estimates:
## mean of x mean of y 
##  93.72414  87.65625
\end{verbatim}

\begin{Shaded}
\begin{Highlighting}[]
\NormalTok{p13 =}\StringTok{ }\KeywordTok{t.test}\NormalTok{(leadData}\OperatorTok{$}\NormalTok{IQ[leadData}\OperatorTok{$}\NormalTok{GROUP}\OperatorTok{==}\StringTok{'1'}\NormalTok{],leadData}\OperatorTok{$}\NormalTok{IQ[leadData}\OperatorTok{$}\NormalTok{GROUP}\OperatorTok{==}\StringTok{'3'}\NormalTok{],}\DataTypeTok{var.equal =}\NormalTok{ F)}
\NormalTok{p13}
\end{Highlighting}
\end{Shaded}

\begin{verbatim}
## 
##  Welch Two Sample t-test
## 
## data:  leadData$IQ[leadData$GROUP == "1"] and leadData$IQ[leadData$GROUP == "3"]
## t = 2.3333, df = 60.024, p-value = 0.023
## alternative hypothesis: true difference in means is not equal to 0
## 95 percent confidence interval:
##   0.9643448 12.5505977
## sample estimates:
## mean of x mean of y 
##  93.72414  86.96667
\end{verbatim}

\begin{Shaded}
\begin{Highlighting}[]
\NormalTok{p23 =}\StringTok{ }\KeywordTok{t.test}\NormalTok{(leadData}\OperatorTok{$}\NormalTok{IQ[leadData}\OperatorTok{$}\NormalTok{GROUP}\OperatorTok{==}\StringTok{'2'}\NormalTok{],leadData}\OperatorTok{$}\NormalTok{IQ[leadData}\OperatorTok{$}\NormalTok{GROUP}\OperatorTok{==}\StringTok{'3'}\NormalTok{],}\DataTypeTok{var.equal =}\NormalTok{ F)}
\NormalTok{p23}
\end{Highlighting}
\end{Shaded}

\begin{verbatim}
## 
##  Welch Two Sample t-test
## 
## data:  leadData$IQ[leadData$GROUP == "2"] and leadData$IQ[leadData$GROUP == "3"]
## t = 0.23761, df = 52.997, p-value = 0.8131
## alternative hypothesis: true difference in means is not equal to 0
## 95 percent confidence interval:
##  -5.131548  6.510715
## sample estimates:
## mean of x mean of y 
##  87.65625  86.96667
\end{verbatim}

The p-value for Group 1 and 2 comparison is \textbf{0.01205}. The
p-value for Group 1 and 3 comparison is \textbf{0.023}. The p-value for
Group 2 and 3 comparison is \textbf{0.8131}.

\textbf{11. What conclusion about the association between lead and IQ
would you draw from the pairwise comparisons of group means? Does it
agree with the conclusion in Q8? (Consider the issue of multiple testing
in your answer.)}

We \textbf{reject the null hypothesis} about the association between
lead and IQ as the p-value is smaller than the level of significance for
two of the pairwise comparisons.

If we consider the issue of multiple testing and apply Bonferroni's
correction on the data, the new level of significance would be 0.05/3 =
0.0167.After comparing with the new significance level, we
\textbf{reject the null hypothesis} as the p-value is small than the new
level of significance for one group.

After applying the Bonferroni correction, the conclusion is not
different from conclusion of Q8.

\textbf{12. Now we wish to compare the 3 group means for males and
females separately. Display some appropriate descriptive statistics for
this analysis.}

\begin{Shaded}
\begin{Highlighting}[]
\NormalTok{m_}\DecValTok{1}\NormalTok{ =}\StringTok{ }\KeywordTok{with}\NormalTok{(leadData,}\KeywordTok{tapply}\NormalTok{(IQ[leadData}\OperatorTok{$}\NormalTok{SEX}\OperatorTok{==}\DecValTok{1}\NormalTok{], GROUP[leadData}\OperatorTok{$}\NormalTok{SEX}\OperatorTok{==}\DecValTok{1}\NormalTok{], mean))}
\NormalTok{s_}\DecValTok{1}\NormalTok{ =}\StringTok{ }\KeywordTok{with}\NormalTok{(leadData,}\KeywordTok{tapply}\NormalTok{(IQ[leadData}\OperatorTok{$}\NormalTok{SEX}\OperatorTok{==}\DecValTok{1}\NormalTok{], GROUP[leadData}\OperatorTok{$}\NormalTok{SEX}\OperatorTok{==}\DecValTok{1}\NormalTok{], sd))}
\NormalTok{n_}\DecValTok{1}\NormalTok{ =}\StringTok{ }\KeywordTok{with}\NormalTok{(leadData,}\KeywordTok{tapply}\NormalTok{(IQ[leadData}\OperatorTok{$}\NormalTok{SEX}\OperatorTok{==}\DecValTok{1}\NormalTok{], GROUP[leadData}\OperatorTok{$}\NormalTok{SEX}\OperatorTok{==}\DecValTok{1}\NormalTok{], length))}
\KeywordTok{data.frame}\NormalTok{(m_}\DecValTok{1}\NormalTok{,s_}\DecValTok{1}\NormalTok{,n_}\DecValTok{1}\NormalTok{,(s_}\DecValTok{1}\OperatorTok{^}\DecValTok{2}\NormalTok{))}
\end{Highlighting}
\end{Shaded}

\begin{verbatim}
##        m_1      s_1 n_1 X.s_1.2.
## 1 92.93478 15.42351  46 237.8845
## 2 90.17647 11.13124  17 123.9044
## 3 86.61538 17.32791  13 300.2564
\end{verbatim}

\begin{Shaded}
\begin{Highlighting}[]
\NormalTok{m_}\DecValTok{2}\NormalTok{ =}\StringTok{ }\KeywordTok{with}\NormalTok{(leadData,}\KeywordTok{tapply}\NormalTok{(IQ[leadData}\OperatorTok{$}\NormalTok{SEX}\OperatorTok{==}\DecValTok{2}\NormalTok{], GROUP[leadData}\OperatorTok{$}\NormalTok{SEX}\OperatorTok{==}\DecValTok{2}\NormalTok{], mean))}
\NormalTok{s_}\DecValTok{2}\NormalTok{ =}\StringTok{ }\KeywordTok{with}\NormalTok{(leadData,}\KeywordTok{tapply}\NormalTok{(IQ[leadData}\OperatorTok{$}\NormalTok{SEX}\OperatorTok{==}\DecValTok{2}\NormalTok{], GROUP[leadData}\OperatorTok{$}\NormalTok{SEX}\OperatorTok{==}\DecValTok{2}\NormalTok{], sd))}
\NormalTok{n_}\DecValTok{2}\NormalTok{ =}\StringTok{ }\KeywordTok{with}\NormalTok{(leadData,}\KeywordTok{tapply}\NormalTok{(IQ[leadData}\OperatorTok{$}\NormalTok{SEX}\OperatorTok{==}\DecValTok{2}\NormalTok{], GROUP[leadData}\OperatorTok{$}\NormalTok{SEX}\OperatorTok{==}\DecValTok{2}\NormalTok{], length))}
\KeywordTok{data.frame}\NormalTok{(m_}\DecValTok{2}\NormalTok{,s_}\DecValTok{2}\NormalTok{,n_}\DecValTok{2}\NormalTok{,(s_}\DecValTok{2}\OperatorTok{^}\DecValTok{2}\NormalTok{))}
\end{Highlighting}
\end{Shaded}

\begin{verbatim}
##        m_2       s_2 n_2  X.s_2.2.
## 1 94.60976 15.877465  41 252.09390
## 2 84.80000  6.471917  15  41.88571
## 3 87.23529  8.898942  17  79.19118
\end{verbatim}

\textbf{13. Perform tests to compare the mean IQ scores in the 3 groups
for males and females separately. Report the p-values from the two
tests.}

\begin{Shaded}
\begin{Highlighting}[]
\KeywordTok{summary}\NormalTok{(}\KeywordTok{aov}\NormalTok{(leadData}\OperatorTok{$}\NormalTok{IQ[leadData}\OperatorTok{$}\NormalTok{SEX}\OperatorTok{==}\DecValTok{1}\NormalTok{]}\OperatorTok{~}\NormalTok{leadData}\OperatorTok{$}\NormalTok{GROUP[leadData}\OperatorTok{$}\NormalTok{SEX}\OperatorTok{==}\DecValTok{1}\NormalTok{]))}
\end{Highlighting}
\end{Shaded}

\begin{verbatim}
##                                   Df Sum Sq Mean Sq F value Pr(>F)
## leadData$GROUP[leadData$SEX == 1]  1    427   427.5   1.942  0.168
## Residuals                         74  16292   220.2
\end{verbatim}

\begin{Shaded}
\begin{Highlighting}[]
\KeywordTok{summary}\NormalTok{(}\KeywordTok{aov}\NormalTok{(leadData}\OperatorTok{$}\NormalTok{IQ[leadData}\OperatorTok{$}\NormalTok{SEX}\OperatorTok{==}\DecValTok{2}\NormalTok{]}\OperatorTok{~}\NormalTok{leadData}\OperatorTok{$}\NormalTok{GROUP[leadData}\OperatorTok{$}\NormalTok{SEX}\OperatorTok{==}\DecValTok{2}\NormalTok{]))}
\end{Highlighting}
\end{Shaded}

\begin{verbatim}
##                                   Df Sum Sq Mean Sq F value Pr(>F)  
## leadData$GROUP[leadData$SEX == 2]  1    922   922.1   5.295 0.0243 *
## Residuals                         71  12366   174.2                 
## ---
## Signif. codes:  0 '***' 0.001 '**' 0.01 '*' 0.05 '.' 0.1 ' ' 1
\end{verbatim}

The p-value from the test to compare the mean IQ scores for males is
\textbf{0.168}. The p-value from the test to compare the mean IQ scores
for females is \textbf{0.0243}.

\textbf{14. What can you conclude about the association between lead and
IQ from these tests? Does it agree with the result in Q8 and Q11?
(Consider multiple testing.)}

From the above questions, we \textbf{do not have enough evidence to
reject the null hypothesis} for males and we \textbf{reject the null
hypothesis} for females.

Thus, We can conclude that there is no association between lead and IQ
for males however there is an association between lead and IQ for
females.

The conclusions agree with the results of Q8 and Q11 for females but not
for males.

\textbf{15. Now perform all 3 pairwise comparisons of groups for males
and females separately. Report the p-values from these tests?}

\begin{Shaded}
\begin{Highlighting}[]
\NormalTok{p12_m =}\StringTok{ }\KeywordTok{t.test}\NormalTok{(leadData}\OperatorTok{$}\NormalTok{IQ[leadData}\OperatorTok{$}\NormalTok{GROUP}\OperatorTok{==}\StringTok{'1'} \OperatorTok{&}\StringTok{ }\NormalTok{leadData}\OperatorTok{$}\NormalTok{SEX}\OperatorTok{==}\DecValTok{1}\NormalTok{],leadData}\OperatorTok{$}\NormalTok{IQ[leadData}\OperatorTok{$}\NormalTok{GROUP}\OperatorTok{==}\StringTok{'2'} \OperatorTok{&}\StringTok{ }\NormalTok{leadData}\OperatorTok{$}\NormalTok{SEX}\OperatorTok{==}\DecValTok{1}\NormalTok{],}\DataTypeTok{var.equal =}\NormalTok{ F)}
\NormalTok{p12_m}
\end{Highlighting}
\end{Shaded}

\begin{verbatim}
## 
##  Welch Two Sample t-test
## 
## data:  leadData$IQ[leadData$GROUP == "1" & leadData$SEX == 1] and leadData$IQ[leadData$GROUP == "2" & leadData$SEX == 1]
## t = 0.78142, df = 39.661, p-value = 0.4392
## alternative hypothesis: true difference in means is not equal to 0
## 95 percent confidence interval:
##  -4.377699  9.894323
## sample estimates:
## mean of x mean of y 
##  92.93478  90.17647
\end{verbatim}

\begin{Shaded}
\begin{Highlighting}[]
\NormalTok{p13_m =}\StringTok{ }\KeywordTok{t.test}\NormalTok{(leadData}\OperatorTok{$}\NormalTok{IQ[leadData}\OperatorTok{$}\NormalTok{GROUP}\OperatorTok{==}\StringTok{'1'} \OperatorTok{&}\StringTok{ }\NormalTok{leadData}\OperatorTok{$}\NormalTok{SEX}\OperatorTok{==}\DecValTok{1}\NormalTok{],leadData}\OperatorTok{$}\NormalTok{IQ[leadData}\OperatorTok{$}\NormalTok{GROUP}\OperatorTok{==}\StringTok{'3'} \OperatorTok{&}\StringTok{ }\NormalTok{leadData}\OperatorTok{$}\NormalTok{SEX}\OperatorTok{==}\DecValTok{1}\NormalTok{],}\DataTypeTok{var.equal =}\NormalTok{ F)}
\NormalTok{p13_m}
\end{Highlighting}
\end{Shaded}

\begin{verbatim}
## 
##  Welch Two Sample t-test
## 
## data:  leadData$IQ[leadData$GROUP == "1" & leadData$SEX == 1] and leadData$IQ[leadData$GROUP == "3" & leadData$SEX == 1]
## t = 1.1886, df = 17.738, p-value = 0.2503
## alternative hypothesis: true difference in means is not equal to 0
## 95 percent confidence interval:
##  -4.862555 17.501351
## sample estimates:
## mean of x mean of y 
##  92.93478  86.61538
\end{verbatim}

\begin{Shaded}
\begin{Highlighting}[]
\NormalTok{p23_m =}\StringTok{ }\KeywordTok{t.test}\NormalTok{(leadData}\OperatorTok{$}\NormalTok{IQ[leadData}\OperatorTok{$}\NormalTok{GROUP}\OperatorTok{==}\StringTok{'2'} \OperatorTok{&}\StringTok{ }\NormalTok{leadData}\OperatorTok{$}\NormalTok{SEX}\OperatorTok{==}\DecValTok{1}\NormalTok{],leadData}\OperatorTok{$}\NormalTok{IQ[leadData}\OperatorTok{$}\NormalTok{GROUP}\OperatorTok{==}\StringTok{'3'} \OperatorTok{&}\StringTok{ }\NormalTok{leadData}\OperatorTok{$}\NormalTok{SEX}\OperatorTok{==}\DecValTok{1}\NormalTok{],}\DataTypeTok{var.equal =}\NormalTok{ F)}
\NormalTok{p23_m}
\end{Highlighting}
\end{Shaded}

\begin{verbatim}
## 
##  Welch Two Sample t-test
## 
## data:  leadData$IQ[leadData$GROUP == "2" & leadData$SEX == 1] and leadData$IQ[leadData$GROUP == "3" & leadData$SEX == 1]
## t = 0.64603, df = 19.325, p-value = 0.5259
## alternative hypothesis: true difference in means is not equal to 0
## 95 percent confidence interval:
##  -7.963105 15.085277
## sample estimates:
## mean of x mean of y 
##  90.17647  86.61538
\end{verbatim}

\begin{Shaded}
\begin{Highlighting}[]
\NormalTok{p12_f =}\StringTok{ }\KeywordTok{t.test}\NormalTok{(leadData}\OperatorTok{$}\NormalTok{IQ[leadData}\OperatorTok{$}\NormalTok{GROUP}\OperatorTok{==}\StringTok{'1'} \OperatorTok{&}\StringTok{ }\NormalTok{leadData}\OperatorTok{$}\NormalTok{SEX}\OperatorTok{==}\DecValTok{2}\NormalTok{],leadData}\OperatorTok{$}\NormalTok{IQ[leadData}\OperatorTok{$}\NormalTok{GROUP}\OperatorTok{==}\StringTok{'2'} \OperatorTok{&}\StringTok{ }\NormalTok{leadData}\OperatorTok{$}\NormalTok{SEX}\OperatorTok{==}\DecValTok{2}\NormalTok{],}\DataTypeTok{var.equal =}\NormalTok{ F)}
\NormalTok{p12_f}
\end{Highlighting}
\end{Shaded}

\begin{verbatim}
## 
##  Welch Two Sample t-test
## 
## data:  leadData$IQ[leadData$GROUP == "1" & leadData$SEX == 2] and leadData$IQ[leadData$GROUP == "2" & leadData$SEX == 2]
## t = 3.2807, df = 53.22, p-value = 0.001831
## alternative hypothesis: true difference in means is not equal to 0
## 95 percent confidence interval:
##   3.812848 15.806664
## sample estimates:
## mean of x mean of y 
##  94.60976  84.80000
\end{verbatim}

\begin{Shaded}
\begin{Highlighting}[]
\NormalTok{p13_f =}\StringTok{ }\KeywordTok{t.test}\NormalTok{(leadData}\OperatorTok{$}\NormalTok{IQ[leadData}\OperatorTok{$}\NormalTok{GROUP}\OperatorTok{==}\StringTok{'1'} \OperatorTok{&}\StringTok{ }\NormalTok{leadData}\OperatorTok{$}\NormalTok{SEX}\OperatorTok{==}\DecValTok{2}\NormalTok{],leadData}\OperatorTok{$}\NormalTok{IQ[leadData}\OperatorTok{$}\NormalTok{GROUP}\OperatorTok{==}\StringTok{'3'} \OperatorTok{&}\StringTok{ }\NormalTok{leadData}\OperatorTok{$}\NormalTok{SEX}\OperatorTok{==}\DecValTok{2}\NormalTok{],}\DataTypeTok{var.equal =}\NormalTok{ F)}
\NormalTok{p13_f}
\end{Highlighting}
\end{Shaded}

\begin{verbatim}
## 
##  Welch Two Sample t-test
## 
## data:  leadData$IQ[leadData$GROUP == "1" & leadData$SEX == 2] and leadData$IQ[leadData$GROUP == "3" & leadData$SEX == 2]
## t = 2.2433, df = 50.748, p-value = 0.02927
## alternative hypothesis: true difference in means is not equal to 0
## 95 percent confidence interval:
##   0.7739536 13.9749703
## sample estimates:
## mean of x mean of y 
##  94.60976  87.23529
\end{verbatim}

\begin{Shaded}
\begin{Highlighting}[]
\NormalTok{p23_f =}\StringTok{ }\KeywordTok{t.test}\NormalTok{(leadData}\OperatorTok{$}\NormalTok{IQ[leadData}\OperatorTok{$}\NormalTok{GROUP}\OperatorTok{==}\StringTok{'2'} \OperatorTok{&}\StringTok{ }\NormalTok{leadData}\OperatorTok{$}\NormalTok{SEX}\OperatorTok{==}\DecValTok{2}\NormalTok{],leadData}\OperatorTok{$}\NormalTok{IQ[leadData}\OperatorTok{$}\NormalTok{GROUP}\OperatorTok{==}\StringTok{'3'} \OperatorTok{&}\StringTok{ }\NormalTok{leadData}\OperatorTok{$}\NormalTok{SEX}\OperatorTok{==}\DecValTok{2}\NormalTok{],}\DataTypeTok{var.equal =}\NormalTok{ F)}
\NormalTok{p23_f}
\end{Highlighting}
\end{Shaded}

\begin{verbatim}
## 
##  Welch Two Sample t-test
## 
## data:  leadData$IQ[leadData$GROUP == "2" & leadData$SEX == 2] and leadData$IQ[leadData$GROUP == "3" & leadData$SEX == 2]
## t = -0.89218, df = 29.016, p-value = 0.3796
## alternative hypothesis: true difference in means is not equal to 0
## 95 percent confidence interval:
##  -8.017810  3.147222
## sample estimates:
## mean of x mean of y 
##  84.80000  87.23529
\end{verbatim}

The p-value from the test to compare the mean IQ scores for Group 1 and
2 males is \textbf{0.4392}. The p-value from the test to compare the
mean IQ scores for Group 1 and 3 males is \textbf{0.2503}. The p-value
from the test to compare the mean IQ scores for Group 2 and 3 males is
\textbf{0.5259}.

The p-value from the test to compare the mean IQ scores for Group 1 and
2 females is \textbf{0.001831}. The p-value from the test to compare the
mean IQ scores for Group 1 and 3 females is \textbf{0.02927}. The
p-value from the test to compare the mean IQ scores for Group 2 and 3
females is \textbf{0.3796}.

\textbf{16. What do you conclude about the association between lead and
IQ from the results in Q15? Does your conclusion change from previous
conclusions made in Q8, Q11 and Q14?}

We \textbf{reject the null hypothesis} for female but \textbf{do not
have evidence to reject the null hypothesis} for male based on the
conclusions from Q15.

The conclusions from the previous questions are as follows:

Q8 -\textgreater{} reject the null hypothesis Q11 -\textgreater{} reject
the null hypothesis Q14 -\textgreater{} reject the null hypothesis for
females and do not reject the null hypothesis for males Q16
-\textgreater{} reject the null hypothesis for females and do not reject
the null hypothesis for males

Thus, we can conclude that there is no association between lead exposure
and IQ values.(from Q8 and Q11) However, when we compare the mean IQ
scores in the 3 groups for males and females separately we see that
there is no association between lead exposure and IQ values for males
but there is an association between lead exposure and IQ values for
female. (from Q14 and Q16).

It can also be noted in all the four questions there is no association
between the lead exposure and mean IQ scores for females.

\end{document}
